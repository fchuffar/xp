\PassOptionsToPackage{unicode=true}{hyperref} % options for packages loaded elsewhere
\PassOptionsToPackage{hyphens}{url}
%
\documentclass[]{article}
\usepackage{lmodern}
\usepackage{amssymb,amsmath}
\usepackage{ifxetex,ifluatex}
\usepackage{fixltx2e} % provides \textsubscript
\ifnum 0\ifxetex 1\fi\ifluatex 1\fi=0 % if pdftex
  \usepackage[T1]{fontenc}
  \usepackage[utf8]{inputenc}
  \usepackage{textcomp} % provides euro and other symbols
\else % if luatex or xelatex
  \usepackage{unicode-math}
  \defaultfontfeatures{Ligatures=TeX,Scale=MatchLowercase}
\fi
% use upquote if available, for straight quotes in verbatim environments
\IfFileExists{upquote.sty}{\usepackage{upquote}}{}
% use microtype if available
\IfFileExists{microtype.sty}{%
\usepackage[]{microtype}
\UseMicrotypeSet[protrusion]{basicmath} % disable protrusion for tt fonts
}{}
\IfFileExists{parskip.sty}{%
\usepackage{parskip}
}{% else
\setlength{\parindent}{0pt}
\setlength{\parskip}{6pt plus 2pt minus 1pt}
}
\usepackage{hyperref}
\hypersetup{
            pdfborder={0 0 0},
            breaklinks=true}
\urlstyle{same}  % don't use monospace font for urls
\usepackage{longtable,booktabs}
% Fix footnotes in tables (requires footnote package)
\IfFileExists{footnote.sty}{\usepackage{footnote}\makesavenoteenv{longtable}}{}
\usepackage{graphicx,grffile}
\makeatletter
\def\maxwidth{\ifdim\Gin@nat@width>\linewidth\linewidth\else\Gin@nat@width\fi}
\def\maxheight{\ifdim\Gin@nat@height>\textheight\textheight\else\Gin@nat@height\fi}
\makeatother
% Scale images if necessary, so that they will not overflow the page
% margins by default, and it is still possible to overwrite the defaults
% using explicit options in \includegraphics[width, height, ...]{}
\setkeys{Gin}{width=\maxwidth,height=\maxheight,keepaspectratio}
\setlength{\emergencystretch}{3em}  % prevent overfull lines
\providecommand{\tightlist}{%
  \setlength{\itemsep}{0pt}\setlength{\parskip}{0pt}}
\setcounter{secnumdepth}{0}
% Redefines (sub)paragraphs to behave more like sections
\ifx\paragraph\undefined\else
\let\oldparagraph\paragraph
\renewcommand{\paragraph}[1]{\oldparagraph{#1}\mbox{}}
\fi
\ifx\subparagraph\undefined\else
\let\oldsubparagraph\subparagraph
\renewcommand{\subparagraph}[1]{\oldsubparagraph{#1}\mbox{}}
\fi

% set default figure placement to htbp
\makeatletter
\def\fps@figure{htbp}
\makeatother


\date{}

\begin{document}

\section{Pour tout savoir sur l'eXtreme~Programming}

par Adrien Machado, version\TeX de \url{http://extremeprogramming.free.fr}

L'eXtreme Programming est une méthode~\textbf{révolutionnaire}~pour la
gestion de projets informatiques.
Basée sur des principes simples, elle permet de venir enfin à bout
des~\textbf{dépassements de délais}
et, donc, de~\textbf{budget}, tout en utilisant des
pratiques~\textbf{humaines}...


\tableofcontents



Sur ce site, vous allez pouvoir retrouver :~

l'\href{http://extremeprogramming.free.fr/page.php?page=historique}{{h}{istorique}}~de
la naissance de la méthode

les~\href{http://extremeprogramming.free.fr/page.php?page=fondements}{{f}{ondements}}~de
la méthode

les~\href{http://extremeprogramming.free.fr/page.php?page=principes}{{p}{rincipes
de mise en oeuvre}}

différents~\href{http://extremeprogramming.free.fr/liens.php}{{l}{iens}}~francophones
très utiles

\hypertarget{historique-de-lextreme-programming}{%
\section{\texorpdfstring{\textbf{Historique de
l'eXtreme~Programming}}{Historique de l'eXtreme~Programming}}\label{historique-de-lextreme-programming}}














\subsection{\texorpdfstring{\textbf{Un constat
alarmant}}{Un constat alarmant}}\label{un-constat-alarmant}

La naissance d'une nouvelle méthode arrive rarement sans raisons. Elle
fait plutôt suite à une période difficile...



\hypertarget{les-3-plaies-du-duxe9veloppement-logiciel}{%
\subsubsection{Les 3 plaies du développement
logiciel}\label{les-3-plaies-du-duxe9veloppement-logiciel}}

D'après une étude américaine menée en 1994 sur 8000 projets, 16 \% n'ont
pas respecté les délais et le budget initial et, pire, 32 \% n'ont
jamais abouti.\\
Ainsi, dans le monde de l'ingénierie logicielle, trois plaies pourtant
bien connues enveniment le développement d'applications.\\
La première concerne le planning difficilement maîtrisé ou impliquant de
nombreuses heures sup. Le deuxième est issue des besoins souvent mal
identifiés ou mal compris en raison d'un cahier des charges mal étudié
ou incomplet.~\\
Tout cela implique des changements au cours du développement, d'autant
plus si le client décide de modifier le produit, après une première
maquette par exemple. Enfin, le dernier problème concerne la livraison
finale du produit qui est parfois buguée.


\hypertarget{la-difficultuxe9-de-ruxe9alisation}{%
\subsubsection{La difficulté de
réalisation}\label{la-difficultuxe9-de-ruxe9alisation}}

Pour cause la conception d'un software n'est pas, contrairement à son
nom, souple en raison de la quantité de travail engendrée
traditionnellement par une modification, même minime du cahier des
charges. Pour exemple, une simple demande exige la remise en question
des tests prévus et une modification parfois très importante de
l'architecture du logiciel. On considère alors que le coût du changement
croit exponentiellement avec le temps.~\\
Avec près de 80\% des projets, toutes tailles confondues, qui ne
respectent pas les contraintes de départ de temps et de budget, les
chefs de projet s'interrogent de plus en plus sur leurs méthodes.~\\
Historiquement, lors du développement d'applications et de projets
e-business, chaque développeur se voit confier une partie du travail, le
tout étant ensuite assemblé. Or, cette organisation fait perdre du
temps, car tous les participants au projet ne travaillent pas de
concert. Par la suite, on essaye plusieurs solutions dont les résultats
ne sont que rarement concluant. La première est de tenter de rattraper
du retard grâce à des heures supplémentaires, mais la qualité en souffre
et le moral des participants aussi. La deuxième est de toute façon de se
conformer aux objectifs qualités coûte que coûte en tentant de rallonger
le délai ou de choisir consciemment de ne pas livrer une partie du
projet. Dans ces cas, le client n'est que rarement content... Enfin, une
dernière solution est de tenter de requérir plus de ressources mais tous
les responsables savent que c'est difficile, et le mot est faible...
Ainsi, soit vous ne les obtenez pas, ce qui vous met en mauvaise
position face à votre responsable, soit vous les obtenez et les coûts
administratifs vous enterrent...\\
A l'approche des dates butoires, on a donc un certain stress qui amène à
prendre certaines décisions qui, le plus souvent, entraînent un
mécontentement du client.~



\hypertarget{le-puxe8re-dxp-kent-beck}{%
\subsection{\texorpdfstring{\textbf{Le père d'XP : Kent
Beck}}{Le père d'XP : Kent Beck}}\label{le-puxe8re-dxp-kent-beck}}

XP est né lors d'une récente vague de création de nouvelles méthodes
parallèlement à l'explosion de l'informatique dans les entreprises et de
l'émergence du freeware.~\\
Lors de cette vague, sont essentiellement apparues des méthodes
qualifiées d'agile dont XP fait partie.



\hypertarget{chrysler-le-berceau-dxp}{%
\subsubsection{Chrysler, le berceau
d'XP}\label{chrysler-le-berceau-dxp}}

Si l'on recherche plus précisément le berceau de la méthode, on tombe
sur un grand projet du groupe Daimler Chrysler datant du milieu des
années 90 et nommé le « Chrysler Comprehensive Compensation » ou « C3 ».
Celui-ci consistait à remettre à jour le système de paie de tout le
personnel gérant plus de 10 000 personnes. Le logiciel d'origine
contenait 2000 classes Smalltalk et 30 000 méthodes dont les modules
transmettaient des données faussent, bien que le programme réponde aux
spécifications et cela malgré 18 mois de développement et des millions
de dollars investis... Quand Kent Beck (un gourou reconnu du monde
Smalltalk) repris le projet en 1996 accompagné de Ward Cunningham, ils
ont préféré dire à la responsable informatique de Chrysler, Sue Unger,
de repartir à zéro plutôt que continuer sur de mauvaises bases. Cette
dernière, en concertation avec ses collègues, décida de se lancer dans
cette « monstrueuse » aventure. Toutefois, elle confia le projet à Beck
qui lui dit pourtant : « Vous ne comprenez pas, je suis consultant, je
ne vais pas sur le terrain », auquel elle a répondu : « Vous ne
comprenez pas, je m'appelle Sue Unger, et vous irez sur le terrain »...



\hypertarget{beck-un-chef-de-projet-novateur}{%
\subsubsection{Beck, un chef de projet
novateur}\label{beck-un-chef-de-projet-novateur}}

Beck sorti alors de sa manche un bon nombre de méthodes qu'il avait
auparavant testées mais il se rendit compte qu'aucune lui permettrait de
venir à bout de ses contraintes et il devait innover... ``C'était un
mélange de préparation soigneuse et de panique », disait-il. Tout
d'abord, le projet devait se dérouler en petites itérations définies par
le client et testées systématiquement pour montrer l'avancement. Le
temps étant compté, réaliser les tests avant le code allait permettre de
réaliser des économies de temps et de personnel puisque l'équipe «
assurance qualité » devenait inutile en raison de la correspondance de
la production avec la demande. Souhaitant encore faire des économies de
temps, la programmation en tandem fut choisie. En effet, à deux le
déboguage est instantané et la rédaction de la documentation est
facilitée : « Si un programmeur parvient à communiquer clairement ses
idées à un autre programmeur travaillant à ses côtés, il y a de grandes
chances pour que celui qui vérifiera le programme deux ans plus tard
n'ait aucun mal à le comprendre ».~~\\
Au fur et à mesure de l'arrivée de ses idées, Beck devenait persuadé du
bien fondé de sa méthode.''Au bout de quinze personnes, c'était devenu
tout à fait concevable``, se souvient-il. Durant l'été 1996, il décida
de baptiser son concept. Puisque les autres méthodes donnaient la
priorité au planning sur la programmation, Beck décida
que''programmation" devrait faire partie du nom. ``J'avais alors besoin
d'un adjectif attrayant, suffisamment descriptif et défendable.'' Il
décida alors de nommer sa méthode ``extreme programming'', après avoir
découvert la puissance des athlètes de l'extrême. ``Je voulais que mes
équipes aient le même sentiment d'invulnérabilité face aux défis à
relever.'' De plus, les pratiques sur lesquelles est bâtie XP sont «
autant de boutons de contrôle qu'il pousse au maximum ». Ainsi, cette
méthode est la réunion cohérente de pratiques peut-être simples mais
portées à un point extrême.\\
Et finalement, sa méthode fut un succès et en 1997, Chrysler avait un
nouveau système de paie, et Beck, assisté de son collègue et ami Ron
Jeffries, écrivait le premier ouvrage d'une série sur la ``méthode XP''
tout en s'appliquant à faire des adeptes\\
C'est donc ce projet qui est à l'origine d'XP et qui a vérifié et testé
ses grands principes, en tirant des enseignements des erreurs de
management précédentes.~



\hypertarget{un-duxe9veloppement-rapide-dans-le-monde}{%
\subsection{\texorpdfstring{\textbf{Un développement rapide dans le
monde}}{Un développement rapide dans le monde}}\label{un-duxe9veloppement-rapide-dans-le-monde}}

La méthode est née réellement en 1997 après la parution du premier livre
de Kent Beck : « eXtreme Programming Explained ».



\hypertarget{un-duxe9veloppement-international-facilituxe9-par-internet}{%
\subsubsection{Un développement international facilité par
internet}\label{un-duxe9veloppement-international-facilituxe9-par-internet}}

La toile a alors joué depuis un grand rôle dans le développement de la
pratique d'XP. Pour preuve, il existe de nombreux forums de discussion
qui lui sont directement consacrés, comme par exemple, pas moins de 71 «
Yahoo Groups » internationaux.\\
Parmi eux, le plus connu et le plus populaire a été créé en décembre
1999 et possède aujourd'hui près de 3400 membres avec une moyenne de
1850 messages postés par mois.~\\
On commence ainsi à trouver XP dans des la plupart des domaines comme
l'électronique (développement des pilotes de matériels), la banque, le
transport et les télécoms.~\\
~\\
La méthode fait aussi une incursion chez certains éditeurs ou dans les
entreprises de secteurs scientifiques et techniques, pour leurs projets
de R\&D. Quelques sociétés de services, dont des agences web, s'y
intéressent également, voyant dans XP un moyen de raccourcir leurs
délais. Un auto-recensement international des projets en XP est
d'ailleurs fait à cette page :
http://c2.com/cgi/wiki?ExtremeProgrammingProjects\\
Voici quelques exemple :\\
- Bayerish landes bank, Credit swiss life, First union national bank\\
- Daimler-Chrysler, Ford Motor Company, BMW\\
- CSEE transport\\
- Nortel\\
- AGF



\hypertarget{xp-sinstalle-progressivement-en-france}{%
\subsubsection{XP s'installe progressivement en
France}\label{xp-sinstalle-progressivement-en-france}}

En France, pour l'heure, les premières mises en oeuvre se limitent
essentiellement à des projets réduits ou non stratégiques. La méthode ne
semble en effet pas assez connue par les directions. Pour preuve, dans
un sondage 19 septembre 2002 du JDNet, plus de la moitié des visiteurs
n'ont jamais entendu parlé de la méthode...\\
Mais la communication va bon train et on voit même apparaître de plus en
plus de formations à XP et des cours dans des universités (Université
d'Artois, de Genève...) et des grandes écoles, notamment à l'INSA Lyon
ou à l'école des mines de Nancy, qui sont tout de même des références
dans le monde informatique.\\
On voit aussi apparaître des sociétés françaises qui se vantent de
maîtriser la méthode afin d'attirer le client...\\
La communauté des XPiens française est bien existante et communique sur
son « Yahoo groupe » : extremeprogramming-France comptant environ 250
membres. Les co-auteurs, Laurent Bossavit et Dominic Williams, du livre
francophone « L'eXtreme Programming : Avec deux études de cas »,
participent d'ailleurs activement aux discussions.\\
XP est donc encore à ses débuts, en terme de connaissance mais est en
phase de devenir une référence mondiale. A suivre...


\hypertarget{situation-par-rapport-aux-autres-muxe9thodes}{%
\subsection{\texorpdfstring{\textbf{Situation par rapport aux autres
méthodes}}{Situation par rapport aux autres méthodes}}\label{situation-par-rapport-aux-autres-muxe9thodes}}

Le but de cette partie est de situer XP par rapport à d'autres méthodes
et en aucun cas de présenter d'autres méthodes.



\hypertarget{xp-une-muxe9thode-agile}{%
\subsubsection{XP : une méthode agile}\label{xp-une-muxe9thode-agile}}

Les méthodes agiles sont des méthodes de gestion de projets
informatiques et de développement qui se positionnent en réaction à des
méthodes traditionnelles jugées trop lourdes. XP fait partie de cette
catégorie dont la première idée a été de réagir face à la bureaucratie
où toute démarche exige une grande quantité de temps. C'est aussi la
plus populaire des méthodes agiles mais on peut aussi citer : ASD
(Adaptative Software Development), SCRUM, Crystal, FDD (Feature Driven
Development) et DSDM (Dynamic System Development Method) qui peuvent
chacune être adaptée à une situation particulière, fonction de la taille
de l'équipe.\\
Depuis le mois de février, les figures de proue des méthodologies de
développement agiles ont constitué l'Agile Alliance.\\
En avril 2002, le PUMA (Proposition pour l'Unification des Méthodes
Agiles) naît poussé par Jean-Pierre Vickoff, consultant formateur,
directeur de projet et architecte de SI, auteur de nombreux ouvrages sur
les méthodes et expert RAD.



\hypertarget{xp-vs-muxe9thodes-traditionnelles-comme-uml}{%
\subsubsection{XP vs Méthodes traditionnelles comme
UML}\label{xp-vs-muxe9thodes-traditionnelles-comme-uml}}

Par rapport aux autres méthodes plus traditionnelles comme RAD et
Unified Process (qui s'appuie sur UML), Jean-Louis Bénard, Directeur
Technique du groupe Business Interactif indique qu'il serait une erreur
de les opposer aux méthodes agiles. « Les méthodes agiles essayent de
formaliser le bon sens et le pragmatisme que des équipes mettent déjà en
oeuvre dans le cadre de projets dits traditionnels ».\\
D'ailleurs, XP peut être vu comme une déclinaison de Unified Process
(UP), en ayant bien en tête cependant que c'est le client qui rédige les
spécifications en XP contrairement à UP où les spécifications sont une
retranscription par l'informaticien de ce qui a été compris. Aussi, les
itérations qui sont un des principes des deux méthodes sont environ 6
fois plus fréquentes en XP.\\
Ainsi, les écoles classiques et « agiles » sont complémentaires.~\\
~\\
UML qui est une méthode de modélisation n'est par essence pas « menacé »
par l'engouement pour XP qui est une méthode de réalisation,
d'organisation.~\\
D'ailleurs, XP utilise le langage universel qu'est UML pour communiquer.
Toutefois, par sa volonté de rapidité, XP reste prudent sur
l'utilisation d'UML. Pour cause, les diagrammes UML sont trop longs à
réaliser alors que la conception en XP étant très courte et réalisée au
fur et à mesure. La documentation par les diagrammes UML est aussi
déconseillée en raison du volume de ces derniers, par rapport aux
quelques pages préconisées dans XP.~




\hypertarget{fondements-de-lextreme-programming}{%
\section{\texorpdfstring{\textbf{Fondements de
l'eXtreme~Programming}}{Fondements de l'eXtreme~Programming}}\label{fondements-de-lextreme-programming}}





\hypertarget{lignes-directrices}{%
\subsection{\texorpdfstring{\textbf{Lignes
directrices}}{Lignes directrices}}\label{lignes-directrices}}

XP est une méthode basée sur des principes simples, que tout le monde
pratique déjà sans le savoir. Mais alors, où est la « révolution » ?


\hypertarget{rendre-moins-lourdes-les-duxe9marches}{%
\subsubsection{Rendre moins lourdes les
démarches}\label{rendre-moins-lourdes-les-duxe9marches}}

Le but d'XP est de trouver des solutions plus ou moins simples, basées
sur des principes éprouvés, pour tenter de diminuer les risques pour
respecter les délais et la qualité du produit commandé.~\\
Cela, en trouvant un compromis équilibré entre "pas de démarche" et
"trop de démarche", tout en respectant le minimum de discipline
nécessaire pour attendre un retour sur investissement en temps et en
effort immédiat et intéressant.\\
Ainsi, on souhaite réduire radicalement la production de documents en se
focalisant sur un code bien commenté plutôt que sur des documentations
techniques très formelles.\\
XP évite donc avec cela les lourdeurs des méthodes classiques et est
donc un habile compromis entre une méthode traditionnelle et le
développement « cow boy », sans règles précises. Par conséquent le
travail des informaticiens est totalement tourné vers la technique où
ils se sentiront vraisemblablement plus à l'aise.


\hypertarget{changer-les-principes}{%
\subsubsection{Changer les principes}\label{changer-les-principes}}

En outre, XP, en tant que méthode agile, se veut adaptative plutôt que
prédictive. C'est à dire qu'à l'inverse des méthodes lourdes qui tentent
de dresser un plan au début du projet, XP indique qu'il vaut mieux agir
sur le court terme. Cela, tout simplement parce qu'il n'existe pas de
projet figé où le prédicat de base ou le contexte ne changent pas du
tout. Dans ce cas, le coût du changement a une croissance logarithmique
en fonction du temps.\\
Le simple fait de ne pas être réticente aux changements permet dans un
premier temps de mieux s'y adapter mais permet aussi une meilleure
relation avec le client et la livraison d'un produit conforme totalement
aux exigences de ce dernier.\\
Enfin, le dernier but d'XP est de se vouloir orienté sur les personnes
plutôt que sur les processus. Alors, le développement tentera d'être une
activité agréable plutôt qu'un cauchemar bureaucratique.


\hypertarget{les-4-valeurs-dxp}{%
\subsection{\texorpdfstring{\textbf{Les 4 valeurs
d'XP}}{Les 4 valeurs d'XP}}\label{les-4-valeurs-dxp}}

Le père d'XP et ses adeptes d'XP définissent quatre valeurs que l'on
doit respecter pour entrer dans le clan, pourtant non fermé, des XPiens.


\hypertarget{communication}{%
\subsubsection{Communication}\label{communication}}

Le manque de communication est dans la vie courant un créateur de
malentendu et cela devient encore plus grave dans un projet, pour tous
les acteurs. Plusieurs pratiques dans XP ont pour finalité de permettre
de se poser les bonnes questions et un partage de l'information entre
acteurs du projet : au sein de l'équipe de développement et entre
développeurs et clients.


\hypertarget{dans-luxe9quipe}{%
\paragraph{Dans l'équipe}\label{dans-luxe9quipe}}

La communication avec XP est une valeur essentielle. Le code étant
propriété de tous, chacun doit tenir au courant les autres de l'état
d'avancement de son travail. Cela permet aussi de demander
éventuellement de l'aide ou de répartir les travaux complexes ou plus
long. Aussi, grâce à cette communication, chacun partage ses
connaissances, ce qui est bénéfique pour l'équipe, le client,
l'entreprise, etc ! Ainsi, se crée une réelle et formidable dynamique de
groupe et un esprit d'équipe solide et efficace.\\
Lors des tests d'évaluations, la communication a lieu aussi avec le chef
de projet puisque ce dernier s'entretient avec son collègue pour savoir
ce qui a été fait et ce qui marche. Par la même occasion, cela permet
éventuellement de redéfinir les tâches en fonction de la charge actuelle
de travail et cela permet aussi d'avoir une discussion amicale avec tout
le monde !\\
La communication écrite n'est pas en reste puisque le code se doit
d'être toujours propre notamment pour permettre une relecture par les
autres.


\hypertarget{avec-le-client}{%
\paragraph{Avec le client}\label{avec-le-client}}

En plus de donner des avantages conventionnels au client, XP permet un
rapprochement du client par rapport aux informaticiens, notamment
lorsque le non-informaticien explique son métier. La communication
s'instaure donc plus facilement entre tous les acteurs et les relations
humaines n'en sont que meilleures.\\
C'est d'ailleurs ce qui ressort d'une interview de Dominic Williams,
chef de projet à CSEE Transport, qui a du fournir une application à
l'entreprise GoldOwner, réalisée par le groupe de réflexion Design-up.
Dans celle-ci, M. Williams explique notamment les bénéfices qu'a apporté
la méthode dans leur relation avec le client.\\
En effet, dans la première phase du projet, le client n'a pas été
intégré au projet alors que dans une deuxième, il a fait partie
intégrante de l'équipe. Selon lui, les rapports et les performances se
sont alors beaucoup améliorés, ainsi que le respect des dates butoires
qui est devenu au fur et à mesure complètement fiable.\\
Enfin, le client a beaucoup apprécié la souplesse que la méthode permet
pour apporter des modifications aux spécifications en cours de projet
(quotidiennement s'il le veut) et la très bonne réactivité pour le
traitement des anomalies. La gestion des priorités est aussi plus
pertinente puisqu'il peut réorienter les développeurs « en direct ».


\hypertarget{feedback}{%
\subsubsection{Feedback}\label{feedback}}

Par le mot feedback, il faut entendre « retours », commentaires, avis...



\hypertarget{du-client}{%
\paragraph{Du client}\label{du-client}}

Tout au long du projet, des retours réguliers du client sont demandés
pour progresser. En effet, avec les livraisons régulières, le client
donne en continue son avis sur les cohérences du produit avec ses
souhaits en terme de fonctionnalité. Le moyen de faire ces feedbacks est
la mise en place des tests unitaires qui détectent les bugs et les
régressions. En découlent des félicitations, des changements et des
améliorations pour coller de plus en plus près à la demande finale. Par
conséquent, les développeurs peuvent programmer sans crainte puisque
toute divergence sera détecter rapidement.


\hypertarget{de-luxe9quipe}{%
\paragraph{De l'équipe}\label{de-luxe9quipe}}

En parallèle au feedback du client, au sein de l'équipe, les
commentaires sont aussi les bien venu. Cela est possible grâce à la
programmation en binôme et puisque chacun peut consulter le code de
l'autre. Là aussi il en découle des congratulations ou des critiques
constructives qui augmentent les compétences du fautif.


\hypertarget{simplicituxe9}{%
\subsubsection{Simplicité}\label{simplicituxe9}}

La vie est déjà assez difficile pour avoir envie de la rendre encore
plus qu'elle ne l'est...


\hypertarget{le-yagni}{%
\paragraph{Le YAGNI}\label{le-yagni}}

Une autre grande idée d'XP est donc de toujours avoir en tête un
principe trivial. Celui-ci porte un nom qui semble complexe mais qui
pourtant à des aspirations très élémentaires : le YAGNI (You're NOT
Gonna Need It). C'est à dire qu'il faut toujours chercher à supprimer
l'inutile pour optimiser au maximum la productivité. Ainsi, un
développeur doit toujours avoir en tête cette phrase : "What is the
simplest thing that could possibly work?" (« Qu'elle est la solution la
plus simple qui peut fonctionner ?").\\
Cette philosophie rejoint l'idée vue précédemment sur la flexibilité
qu'apporte XP aux changements. Cela car, « il vaut mieux faire simple
aujourd'hui, quitte à changer demain, plutôt que de faire tout de suite
compliqué sans être absolument certain de l'utilité de ce que l'on
développe ».\\
En conséquence, tout au long du développement, la vie du développeur
n'est pas de tenter de rechercher une solution exceptionnelle pour les
sujets considérés mais son but est plutôt d'évoluer le plus vite
possible. Aussi, avec cette souplesse, le client peut demander des
modifications sans que le développeur ne soit réticent à l'appliquer,
sans non plus d'augmentation considérable des délais.


\hypertarget{pour-faire-des-uxe9conomies}{%
\paragraph{Pour faire des économies}\label{pour-faire-des-uxe9conomies}}

De ce point de vue, XP fait donc le pari que la simplicité d'aujourd'hui
revient moins cher et le surcoût éventuel dans le futur est compensé par
le fait que, à propos de la solution compliquée, "vous n'en aurez
probablement pas besoin". Donc, en moyenne, mieux vaut faire simple.\\
D'ailleurs cette simplicité est aussi essentielle pour que le code
puisse être repris instantanément par n'importe quel développeur pour le
compléter ou le changer.


\hypertarget{pruxe9cisions}{%
\paragraph{Précisions}\label{pruxe9cisions}}

Attention toutefois, "le plus simple possible" ne signifie pas
"simpliste", il ne faut pas se tromper sur ce but ! Par exemple, il ne
faut pas se laisser tenter par la simplicité de ne pas reprendre une
classe par exemple pour éviter de s'embêter à la relier et la comprendre
(ce qui doit d'ailleurs être chose aisée !)... Toute duplication de code
est bien sûr interdite !\\
Enfin, le client se doit aussi d'adhérer à ce principe pour ne pas
demander des choses complexes qui ne seront jamais utilisées !


\hypertarget{courage}{%
\subsubsection{Courage}\label{courage}}

Les acteurs d'un projet qui décident d'utiliser XP, même si la méthode a
de grandes qualités, doivent avoir du courage.~


\hypertarget{avoir-confiance-en-xp}{%
\paragraph{Avoir confiance en XP}\label{avoir-confiance-en-xp}}

Le client doit avoir le courage de donner des priorités à ses besoins,
de dire que tel besoin n'est finalement pas très clair...~\\
Le responsable doit avoir le courage de commencer le projet avant que
tout ait été précisé clairement sur un document contractuel. Et même en
amont, il doit faire preuve de courage en adoptant une méthode nouvelle,
surtout sur des projets stratégiques.


\hypertarget{savoir-jeter-pour-repartir-sur-de-bonnes-bases}{%
\paragraph{Savoir jeter pour repartir sur de bonnes
bases}\label{savoir-jeter-pour-repartir-sur-de-bonnes-bases}}

De même pour le développeur qui doit s'obliger à jeter du code qui sent
mauvais... Le code a une odeur, quelquefois il faut le changer et tout
recommencer. Simplement, XP demande le courage de regarder le
développement en face, sans tabou... De plus, la mise en place
d'itérations très courtes implique une possibilité de reconsidération
régulière de son travail ce qui serait susceptible de décourager
certains, voire la plupart !\\
Enfin, la transparence vis à vis du client et de l'équipe
(particulièrement des binômes) demande du courage car chacun peut voir
les incompétences de l'autre...


\hypertarget{principes-de-lextreme-programming}{%
\section{\texorpdfstring{\textbf{Principes de
l'eXtreme~Programming}}{Principes de l'eXtreme~Programming}}\label{principes-de-lextreme-programming}}









\hypertarget{spuxe9cifications-ituxe9ratives-par-le-client}{%
\subsection{\texorpdfstring{\textbf{Spécifications itératives par le
client}}{Spécifications itératives par le client}}\label{spuxe9cifications-ituxe9ratives-par-le-client}}

La mise en œuvre d'XP est basée sur des principes que l'on retrouve tout
au long de la vie du projet et qui régissent le cycle de vie de ce
dernier.\\
Avec des méthodes classiques, plus le temps passe, et plus les
réalisations semblent s'éloigner du souhaits (changeant !) du client.
Alors le mieux est de réaliser de nombreuses itérations...


\hypertarget{diviser-pour-mieux-ruxe9gner}{%
\subsubsection{Diviser pour mieux
régner}\label{diviser-pour-mieux-ruxe9gner}}

Cette expression va vraiment dans la philosophie d'XP, si bien sur on ne
parle pas le sens péjoratif qui est de diviser l'équipe mais de se
séparer pour arriver à de meilleurs résultats !




\hypertarget{des-spuxe9cifications-ruxe9guliuxe8res}{%
\paragraph{Des spécifications
régulières}\label{des-spuxe9cifications-ruxe9guliuxe8res}}

Tout projet commence obligatoirement par les précisions de la demande du
client : les spécifications. Cependant, avec XP, cette première phase ne
prend pas le temps généralement constaté. En effet, ne dit-on pas «
diviser pour mieux régner » ? C'est tout à fait la façon de penser
d'XP.~\\
Plutôt que de tout spécifier au début du projet, le client (on entend
par client une personne ayant une vision d'utilisateur final et
habilitée à définir les besoins du projet et leurs priorités) va
expliquer aux développeurs la première fonctionnalité qui est pour lui
la plus importante. Cela se fait lors de réunions nommées les « planning
game » où la discussion entre les développeurs et le client est très
importante. On y discute des spécifications, des solutions et du
planning. Ensuite, les informaticiens s'attachent à implémenter cette
première demande dont le temps de développement ne devra dépasser une à
deux semaines.~




\hypertarget{de-nombreuses-ituxe9rations}{%
\paragraph{De nombreuses itérations}\label{de-nombreuses-ituxe9rations}}

Chaque partie du projet, qui devient très itératif, est ainsi divisée en
sous-modules qui doivent être opérationnels au bout d'un intervalle de
temps prédéfini relativement court.~\\
A la fin de chaque itération, une mini-livraison est faite permettant au
client de déjà « s'amuser » avec le produit, de le tester et de se
rendre compte du résultat final. Grâce à ces premiers essais, le client
peut demander des améliorations, des réorientations plus ou moins
profonde au développeur qui a une grande réactivité puisque tout est
encore « chaud » dans sa tête. Une fois que la fonctionnalité est
validée, on passe à une autre période de spécification, etc.~\\
Avoir plusieurs changements au fur et à mesure est d'ailleurs une chose
préférable à un grand chamboulement !\\
Dans la pratique, un délai de 3 semaines est généralement choisi par
l'équipe pour une fonctionnalité à tester. Ces délais sont bien sûr
exhaustifs mais sont généralement considérés comme optimaux par l'équipe
et le client.




\hypertarget{duxe9finition-des-besoins-par-des-user-stories}{%
\subsubsection{Définition des besoins par des
user-stories}\label{duxe9finition-des-besoins-par-des-user-stories}}

Traduit en français, user-stories signifie « histoires de l'utilisateur
» et c'est bien ce que ça représente : récit de ce que veut
l'utilisateur.




\hypertarget{duxe9finition}{%
\paragraph{Définition}\label{duxe9finition}}

On demande donc au client de définir les spécifications en petites
histoires testables en 1 à 2 semaines. Or, n'arrive-t-il jamais que le
client de sache pas comment faire ? Après tout, chacun sa spécialité !
En faite, la question ne se pose pas car dans XP, le client utilise des
« user stories » pour définir ses besoins. Comme leurs noms l'indiquent,
ces fonctionnalités nécessaires à l'utilisateur sont décrites à la façon
d'une bande dessinée (contenu simple et imagé), sur des fiches bristol
format A5. Elles sont donc courtes et n'abordent pas les considérations
techniques pour rendre la production la plus intuitive possible. On doit
utiliser des métaphores pour que tout le monde puisse bien se
comprendre. Tout doit rester bien concret avec XP !\\
~\\
Inévitablement, les besoins ne sont pas suffisamment précis. Le
développeur est donc dans l'obligation de dialoguer avec le client pour
obtenir les détails ce qui favorise encore une fois l'établissement
d'une réelle communication.




\hypertarget{points-forts-et-faiblesses}{%
\paragraph{Points forts et
faiblesses}\label{points-forts-et-faiblesses}}

L'un des avantages de cette technique est que l'utilisateur écrit
vraiment ses besoins, alors que dans une approche plus classique le
développeur (ou l'ingénieur d'affaire ou le chef de projet) retranscrit
plus ou moins bien ce qu'il a compris.\\
Ainsi, le résultat est fondamentalement différent. En effet, les user
stories sont vraiment une technique d'expression des besoins alors que
les cas d'utilisation, demandés par les méthodes classiques (UML), se
cantonnent souvent à une translation des besoins. De plus, l'implication
du client est également sans commune mesure. Dans XP, il est
effectivement partie prenante dans le pilotage du projet, itérations
après itérations.\\
Toutefois, le niveau de détail d'une « user story » est inférieur à
celui d'un cas d'utilisation UML. Pour cause, elle n'est constituée que
d'un titre, d'une ou deux phrases d'explication du fonctionnement voulu
et d'un schéma, sans aborder les conditions d'erreurs et les traitements
associés.~\\
Aussi, XP se voulant très itératif, son contenu ne pourra excéder un
cycle d'itération ou dans le cas contraire, l'user story devra être
découpée. A l'opposé, une user story trop légère sera fusionnée avec un
autre bristol.




\hypertarget{les-pruxe9visions-duxe9tailluxe9es}{%
\subsubsection{Les prévisions
détaillées}\label{les-pruxe9visions-duxe9tailluxe9es}}

Le rôle d'un chef de projet est souvent décrit comme la personne qui
planifie. Qu'en est-il de cette problématique avec XP ?




\hypertarget{des-estimations-pruxe9cises}{%
\paragraph{Des estimations précises}\label{des-estimations-pruxe9cises}}

Grâce à ces jalons fréquents, on peut se permettre de prévoir avec une
plus ou moins grande précision les délais et la date de fin du projet :
chaque fiche correspond à une itération dont la durée est
approximativement connue. Cette durée est composée d'unités arbitraires
(XP units ou gummy bear) dont la valeur est discutée avec le client et
dont le nombre est défini par les développeurs. En effet, seuls ces
derniers connaissent précisément le temps qui leur est nécessaire, en
fonction de leurs compétences et de leurs expériences, pour aboutir au
résultat décrit dans la fiche. D'ailleurs, pour se décider, le
développeur peut utiliser son PC pour faire de l'expérimentation, ce qui
est nommé « spike solutions ».\\
De même, on peut prévoir les coûts humains puisque chaque user story
engendre un cycle de développement d'une ou deux paires de programmeurs.



\hypertarget{des-mises-uxe0-jour-ruxe9guliuxe8res}{%
\paragraph{Des mises à jour
régulières}\label{des-mises-uxe0-jour-ruxe9guliuxe8res}}

A noter que ces prévisions peuvent, bien sûr, être re-négociées et
re-considérées puisque le projet est revu à chaque itération. Cela, pour
deux raisons : la première est que tout ne se passe jamais comme prévu
(une équipe de développeur peut rester bloqué sur un problème qui ne
semblait pas complexe par exemple). La deuxième vient du fait qu'avec XP
le client peut se permettre de changer, compléter ou de supprimer sa
demande.~\\
Quoi qu'il en soit, plus on arrivera vers la fin du projet et plus les
estimations seront précises, notamment avec l'augmentation des
compétences des programmeurs leur permettant d'affiner leurs
estimations.\\
Les prévisions reprécisées régulièrement, à chaque itération, permettent
de donner aux directions des chiffres précis et en temps réel, pour
anticiper des dérapages éventuels et diminuer les risques de
dépassement. En connaissance de cause, les responsables pourront ainsi
ajouter, si nécessaire, des ressources pour respecter les délais,
redéfinir les objectifs ou, même, décider de stopper le projet, s'il
semble s'avérer irréalisable, avant que les pertes ne soient trop
grandes.~




\hypertarget{les-tests-uxe9crits-avant-le-programme}{%
\subsection{\texorpdfstring{\textbf{Les tests écrits avant le
programme}}{Les tests écrits avant le programme}}\label{les-tests-uxe9crits-avant-le-programme}}

Après avoir obtenu des user stories précises, les développeurs entrent
en jeu plus activement. Leur première activité après les spécifications,
n'est pas de coder les fonctionnalités mais les tests. « Ecrire les
tests puis coder : si vous ne le faites pas, vous n'êtes pas un Extreme
Programmer » affirme Kent Beck.\\
Avec le principe de ne travailler que par itérations courtes, le
processus de test se voit pour cela aussi modifié.




\hypertarget{ecrire-les-tests-avant-tout}{%
\subsubsection{Ecrire les tests avant
tout}\label{ecrire-les-tests-avant-tout}}

XP se veut totalement novateur dans le domaine des tests. En effet, les
tests jouent un rôle fondamental dans XP à tel point que leur importance
est comparable à celle de la programmation elle-même.\\
D'ailleurs, on considère que les tests sont une partie intégrante de la
documentation puisque que la programmation de conditions de tests est
une indication directe sur la façon d'utiliser l'élément testé.




\hypertarget{des-tests-uxe9crits-tuxf4t-et-automatisuxe9s}{%
\paragraph{Des tests écrits tôt et
automatisés}\label{des-tests-uxe9crits-tuxf4t-et-automatisuxe9s}}

Avec XP, dans la pratique, la mise en place des tests demande une
rigueur importante. Le but est de réaliser à chaque itération des tests
fonctionnels définis par les souhaits du client. Ainsi, le développeur
écrit donc le programme de test unitaire à partir de souhaits du client
et cela, AVANT le code de l'application.~\\
~\\
D'écrire les tests avant, permet de ne rien oublier puisque les
spécifications viennent d'être exposées par le client. De plus, on code
les tests, qui sont garant de la qualité du produit final, quand on
n'est pas encore lassé du développement, avant les fatigantes séances de
développement...\\
~\\
Une telle rigueur n'est que très rarement demandée dans les méthodes de
gestion de projets informatiques. La phase de tests y est souvent
complètement disjointe de la programmation et même parfois faite par une
toute autre équipe. Or, cela entraîne forcément des pertes de temps
puisque le codeur doit prendre le temps de comprendre l'application et
parce que le développeur doit se remettre en tête le code à débugger
après cette phase de test plus ou moins longue !\\
~\\
Avec XP, les tests sont écrits au début mais sont aussi caractérisés par
leur automatisme ! Ainsi, lancer un test ne revient pas à plusieurs
tâches laborieuses à faire les unes à la suite des autres. Tout est
regroupé dans une fonction « main » qu'il suffit de lancer. Si tout se
passe bien, on le sait tout de suite et si un problème survient, nos
propres commentaires permettent de savoir d'où vient l'erreur.~\\
En tout cas, il faut noter que le résultat ne peut être que dichotomique
: soit le module fonctionne parfaitement bien dans sa globalité et on
peut continuer, soit il faut corriger les problèmes avant de passer à
autre chose.~




\hypertarget{les-2-types-de-tests}{%
\paragraph{Les 2 types de tests}\label{les-2-types-de-tests}}

XP définit deux types de tests. Les premiers sont les tests de recette
ou tests fonctionnels qui sont rédigés par le client, à partir d'un
langage formel si possible. Comme leur nom l'indique, ces tests
(Acceptance Tests) servent à valider le travail du prestataire, comme on
peut le voir dans les projets classiques avec un cahier de recette.
Ainsi, il s'agit tout simplement de vérifier que tout fonctionne
correctement et est conforme à la demande, avant de donner le dernier
chèque... Mais, avec XP, le client définissant au fur et à mesure ses
besoins aux développeurs, il n'y a généralement pas de grandes surprises
lors de ces tests...\\
~\\
Le deuxième type de test que recense XP est le test unitaire (Unit
Testing). On appelle généralement test unitaire, tout test écrit par
l'un des programmeurs dans le but de s'assurer du bon fonctionnement de
son programme. Avec XP, c'est la même signification sauf qu'ils sont
plus nombreux et plus fréquents. Ainsi, au contraire du test de recette,
son but n'est pas de satisfaire le client directement mais permet de
vérifier la qualité du développement. Ces tests sont, bien sûr, écrits
comme les autres avant le codage et sont exécutés après chaque phase de
codage.\\
Ils doivent tester généralement une itération et leur durée d'exécution
est comprise entre 1 et 10 minutes. Au-delà, ils doivent être
découpés.\\
~\\
Ces tests sont les plus importants car ils garantissent la stabilité du
produit et cela, dès le début, pour éviter de partir sur des bases
bancales. Pour cause, ils garantissent la qualité du produit mais
incitent aussi le développeur à réfléchir sérieusement sur la
conception, puisqu'ils sont réalisés avant le code. Ils peuvent être
alors considérés comme plus « contractuels » que les tests de recette
qui découlent de ceux-là.\\
~\\
Pour écrire ces tests, la démarche est la suivante : on commence par
écrire le test puis on l'exécute après la compilation qui garantie la
logique du code. Puisque le test est codé avant le programme, le test
échouera. Une fois que le code de l'application est écrit, il devra par
contre réussir.~\\
Une modification du code (refactoring) ne devra en aucun faire échouer
les tests qui garantissent donc, dès le début, la conformité du
développement. Par contre, un changement dans le caractère fonctionnel
du code, rappellera au développeur que ce changement est peut-être
inutile car non demandé par le client.~




\hypertarget{un-produit-fiable-et-toujours-opuxe9rationnel}{%
\subsubsection{Un produit fiable et toujours
opérationnel}\label{un-produit-fiable-et-toujours-opuxe9rationnel}}

Le rêve diront en cœur développeurs et clients... Avec XP, le rêve peut
devenir réalité !




\hypertarget{des-tests-conformes-uxe0-la-demande-du-client}{%
\paragraph{Des tests conformes à la demande du
client}\label{des-tests-conformes-uxe0-la-demande-du-client}}

Les tests systématiques assurent que le système en cours de création est
constamment opérationnel puisqu'un dysfonctionnement provoqué par une
évolution du code source est tout de suite détecté et donc rapidement
corrigé.\\
~\\
De ce fait, le client peut toujours connaître l'avancement exact du
projet et surtout obtenir des livraisons régulièrement.~~\\
Aussi, la livraison, qui peut se faire donc dès la fin du développement,
est donc celle d'un produit totalement fonctionnel et sans aucun bug.\\
~\\
En outre, les tests permettent d'analyser les résultats, déterminer ce
qui vous ralentit, et améliorer les procédures pour aller plus vite,
tout cela, presque en « live ».~\\
Les tests étant écrits avant le codage, une fois que la partie
programmation est finie, on peut tester tout de suite le module.~\\
De ce fait, on peut obtenir très rapidement un feed-back du client qui
validera la partie ou demandera tout de suite des modifications, tant
que tout est « tout chaud » dans la tête du développeur.\\
On a donc à chaque itération, un raffinage instantané des besoins
clients d'autant plus que chacune des modifications demandées est aussi
testée jusqu'à une satisfaction totale du client.\\
~\\
Par conséquent, avec ces tests écrits avant le début de la programmation
du logiciel, on obtient obligatoirement du code conforme à la demande.
On évite aussi les oublie car on ne programme que ce qui va être testé,
c'est à dire tout ce qui est demandé.~




\hypertarget{des-tests-optimaux-pour-une-meilleure-conception}{%
\paragraph{Des tests optimaux pour une meilleure
conception}\label{des-tests-optimaux-pour-une-meilleure-conception}}

Ecrire les tests d'une classe avant même d'écrire la classe permet aussi
de réfléchir à son fonctionnement dans sa globalité et à ses interfaces.
Avec ce recul, puisqu'on ne se précipite pas tête baissée dans la
technique, on peut remarquer, dès cette phase de conception, les oublis
ou manques éventuels des user stories.\\
~\\
D'avoir les tests avant la fin du développement permet aussi de tester
régulièrement le code créé, d'un point de vu syntaxique, sémantique mais
aussi logique. En effet, pour les langages assembleurs seule la syntaxe
était vérifiée, les compilateurs C s'occupe aussi de la sémantique et
ceux des langages objets vont encore plus loin : « cet objet est-il bien
utilisé ? » , « Cette méthode est-elle bien appelée ? ». Avec les
bibliothèques de tests Junit (de java) et CppUnit (de c++), la
philosophie des tests performants est encore plus intégrée au langage
lui-même !\\
~\\
Mais les tests présents dans les compilateurs ne peuvent aller plus loin
et les fonctionnalités ne peuvent être testées. C'est pourquoi d'écrire
les tests au début, permet d'avoir systématiquement et automatiquement,
après chaque phase de développement, l'assurance d'avoir un module bien
conçu et totalement correct.\\
~\\
D'ailleurs, on en arrive à se demander si ce n'est pas grâce aux tests
que l'on arrive à programmer mieux et plus rapidement. Pour cause, on
possède un garde-fou : si on fait quelque chose de faux, le filet de
sécurité des tests intégrés au langage prévient qu'il y a un problème et
montre même où il réside...




\hypertarget{programmation}{%
\subsection{\texorpdfstring{\textbf{Programmation}}{Programmation}}\label{programmation}}

Dans une méthode de gestion de projet de développement logiciel, on
s'attache souvent à respecter beaucoup de règle d'organisation mais on
s'occupe beaucoup moins de la matière première du projet : le code
source.~




\hypertarget{un-code-propre-et-efficace}{%
\subsubsection{Un code propre et
efficace}\label{un-code-propre-et-efficace}}

Tout comme le voulait nos Mamans : « je veux que ta chambre soir propre
» ! C'est aussi ce que nous disent les XPiens, au sujet du code.




\hypertarget{programmer-simple...}{%
\paragraph{Programmer simple...}\label{programmer-simple...}}

Pour développer avec XP, il faut respecter des règles précises qui ont
beaucoup d'intérêts.~\\
La conception doit être la plus simple possible (Simple Design) surtout,
au début, puisque le code est successible d'être complètement modifié
par la suite, en cas de changement de besoins.\\
Le code créé doit être composé de petites classes et de courtes
méthodes. La relecture du code doit être la plus aisée possible. Cela
passe aussi par l'utilisation de noms explicites pour les variables, les
méthodes, etc. Elémentaire pourrait-on penser, et bien oui ! XP utilise
bien des principes de bases « poussés à fond »!\\
La relecture passe aussi par la présence obligatoire de commentaires
dans le code qui sont dans la pratique souvent bâclés. Il faut aussi
penser à remettre à jour ces commentaires au même rythme que le code ce
qui n'est malheureusement pas toujours le cas. Attention toutefois à ne
pas inonder le code de commentaire. On doit insérer dans le code le
strict nécessaire.\\
Aucune duplication de code ne doit exister, conformément au principe DRY
(Don't Repeat Yourself) de « The Pragmatic Programmer ».\\
Grâce à l'application de ces règles, on obtient un code homogène pour
toute l'application. Et pourtant, l'équipe est composée d'une dizaine de
développeurs de générations différentes !~




\hypertarget{du-code-propre-gruxe2ce-au-refactoring}{%
\paragraph{Du code propre grâce au
refactoring}\label{du-code-propre-gruxe2ce-au-refactoring}}

XP va plus loin encore dans la clarté du code. En effet, des séances de
« refactoring » doivent avoir lieu régulièrement et au moins
quotidiennement. Ce terme a été défini et décrit très en détail par
Martin Fowler dans son ouvrage « Refactoring, Improving the Design of
Existing Code » qui n'est pas disponible en version francophone. Sa
définition est « toute modification d'un programme permettant d'en
améliorer la structure interne sans en modifier le comportement externe
».\\
Pendant ces périodes, le développeur permet de prendre un peu de recul
sur le code et de revoir d'un coté l'architecture du programme et de
l'autre, la simplicité du code. Par exemple, il s'agit de décomposer les
algorithmes en entités très courtes à placer dans des méthodes
distinctes, de donner des noms clairs aux variables et méthodes et
surtout de positionner correctement dans l'espace, les blocs formés par
les boucles et les conditions.\\
Ainsi, un « extrême développeur » se doit d'améliorer fréquemment le
code existant afin qu'il puisse mieux s'adapter aux exigences du
lendemain.~




\hypertarget{lorganisation-du-duxe9veloppement}{%
\subsubsection{L'organisation du
développement}\label{lorganisation-du-duxe9veloppement}}

Le développement est loin d'être une simple tâche de dactylographie.
Cette activité se doit donc d'être bien organisée, et d'autant plus avec
XP.




\hypertarget{pair-programming}{%
\paragraph{Pair Programming}\label{pair-programming}}

Concernant le développement, une caractéristique d'XP, qui est aussi la
plus controversée, est la programmation par binôme, c'est à dire, deux
développeurs pour une machine. Et le but n'est pas dans cette démarche
de faire des économies de matériel informatique...~\\
Les règles pour réaliser cette pratique sont strictes. Il ne s'agit pas
de permettre à l'un de se reposer pendant que l'autre travail. Chacun
programme sa partie de code, dont il reste propriétaire. La
programmation se fait alors à tour de rôle et l'échange du clavier est
régulier, soit à la fin de chaque classe ou quand la fatigue commence à
trop peser.~\\
Dans la programmation en binôme, la personne qui ne tape pas au clavier
peut réfléchir à la suite, à l'architecture globale de l'application,
vérifier la frappe et surtout commenter pour faire évoluer le code. Une
personne produit donc le code (le « driver ») tandis que l'autre prend
du recul pour vérifier la cohérence globale et réfléchir à la conception
générale. Le « partner » garde aussi à porté de main les règles d'XP !
Cela permet aussi d'avoir une autre personne qui sait ce qui a été fait
et qui est capable de continuer au cas où (accident ou vacances).\\
Aussi, il est important que les binômes changent régulièrement, c'est à
dire à chaque nouvelle itération d'une à deux jours. Ainsi, chacun
travail avec toutes les personnes de l'équipe, ce qui permet forcément
de un rapprochement et permet aussi un partage de savoir entre les
développeurs. Aussi, si un binôme reste bloqué, leur discussion peut
être entendue par une autre personne de l'open workspace (grand bureau
partagé par plusieurs personnes), qui pourra venir apporter son aide.\\
La première critique qu'on pourrait faire à ce principe est la perte de
productivité globale. Pourtant, ce fondement fait la force d'XP car,
dans la pratique, le rendement horaire s'approche de 1. Pour cause, la
fatigue de l'un est compensée par l'énergie de l'autre et ils
s'auto-disciplinent. Aussi, de grandes économies sont réalisées pour la
phase de débuggage puisqu'on s'offre une relecture constante du code.~




\hypertarget{des-duxe9veloppeurs-heureux}{%
\paragraph{Des développeurs heureux}\label{des-duxe9veloppeurs-heureux}}

XP se veut une méthode humaine et plutôt orientée vers le conforme des
développeur et cela pou plusieurs raison. La première est que la
communication est grande et notamment grâce aux « stand-up meeting »,
réunions debout (pour s'assurer qu'elles ne s'éternisent pas)
quotidiennes qui permettent d'exposer la situation de chacun, les
éventuelles demandes d'aide... Aussi, le développeur a la possibilité de
choisir l'itération qu'il souhaite traiter, à condition bien sûr que le
sérieux de l'équipe soit assez grand pour ne pas laisser traîner une
tâche lassante qui peut ralentir l'avancée globale du projet. Ainsi, les
développeurs ne se sentent pas frustrés par l'autoritarisme d'un boss,
qui ne fait alors plus que motiver son équipe...\\
Chacun ne peut aussi refuser d'apporter son aide à un collègue !\\
Grâce à tout cela, naît aussi un réel esprit d'équipe et réelle
dynamique de groupe. Les développeur joue pour gagner (et non pour
participer) et ce la fonctionne !~\\
Dans la pratique, on a pu aussi constater que les développeurs étaient
plus motivés, heureux et compétents.\\
Enfin, XP préconise de ne pas réaliser d'heures supplémentaires (si si
!), en tout cas, jamais deux semaines de suite. Une semaine de travail
ne devra qu'exceptionnellement dépasser 40 heures (comme pour les
"40-hour week" des Etats-Unis). En effet, une telle surcharge indique un
problème structurel, de planning, plutôt qu'un problème de productivité.
De plus, la réalisation d' « heures sup » ne peut qu'avoir un effet
néfaste sur la productivité, la motivation et surtout sur la qualité du
code !




\hypertarget{documentation}{%
\subsection{\texorpdfstring{\textbf{Documentation}}{Documentation}}\label{documentation}}

Avec les commentaires du code pourtant complets, il faut quand même
prévoir une petite documentation. En effet, XP n'est pas
anti-documentation, par contre, la méthode se veut très vigilante la
quantité des documents.




\hypertarget{uniquement-deux-types-de-documents}{%
\subsubsection{Uniquement deux types de
documents}\label{uniquement-deux-types-de-documents}}

Par conséquent, il faut bien se demander quels documents sont réellement
utiles. A ce sujet, il faut distinguer deux types de documents
correspondant aux deux types de destinataires, les informaticiens et le
client. Pour ces derniers, ils ne peuvent être réduits voir supprimer
puisqu'ils apparaîtront dans le décompte final de la facture.\\
Par contre, une importante réflexion doit avoir lieu avant de commencer
toute rédaction. En effet, il faut bien avoir en tête que de bons
commentaires devraient suffisant et surtout que le travail en équipe et
plus particulièrement par deux tels que le définit XP, ne devraient pas
engendrer de documentation complémentaire.\\
~\\
Aussi, la création de documentation ne permet en aucun cas d'augmenter
la vitesse de programmation, de rendre le système plus fiable ou même de
le rendre plus facile à maintenir. Les seuls intérêts qui pourraient
être envisagé seraient une éventuelle aide à la relecture par une autre
équipe, une éventuelle lecture pour les clients ou enfin un support pour
la publication d'articles. Or, ces cas ne semblent que guère fréquents,
voir exceptionnels.




\hypertarget{des-formes-bien-pruxe9cises...}{%
\subsubsection{Des formes bien
précises...}\label{des-formes-bien-pruxe9cises...}}

Toutefois, il peut arriver qu'une documentation soit nécessaire dans un
cas où la globalité du projet serait difficile à voir. Mais dans ce cas,
il faut bien s'interroger sur l'utilité effective de ce document qui ne
doit être en aucun cas être redondant par rapport à un éventuel dossier
existant. Aussi, le document final ne devra en aucun cas excéder dix
pages et devra mixer textuel, diagrammes (UML par exemple mais épurés)
et surtout rediriger vers les fichiers sources (le code) et leurs «
javadoc » (documentation au format HTML créé automatiquement par java, à
partir des commentaires du code). Notons bien qu'on pourra s'aider des
tests qui mettent en avant les fonctionnalités essentielles de
l'application.\\
Pour conclure, concernant les spécifications pour la documentation, on
retrouve bien le principe YAGNI de XP avec la suppression de tout
travail inutile. Encore une fois, ce principe tout simple est une grande
innovation puisque dans les autres méthodes, une documentation de
plusieurs dizaines de pages est demandée alors qu'elle ne sera
probablement jamais lue par personne.




\hypertarget{intuxe9gration---livraison}{%
\subsection{\texorpdfstring{\textbf{Intégration -
Livraison}}{Intégration - Livraison}}\label{intuxe9gration---livraison}}

Le fait qu'XP ne permettent pas réellement de prévoir à long terme les
échéances, ne pose en aucun cas un problème pour les livraisons.\\
En effet, les intégrations des classes développées sont faites plusieurs
fois par jours ou, au moins tous les soirs ! Les tests sont alors lancés
pour vérifier la cohérence et la non-régression du produit.\\
Cela à l'avantage d'avoir un produit fonctionnel toujours à jour, ce qui
est permit par le fait que les tests sont écrits auparavant. Ainsi, un «
eXtreme Programmer » ne sera que très rarement contraint de passer des
week-ends ou des nuits à développer pour respecter une deadline. Cela
permet aussi de permettre à l'équipe et au client de connaître
précisément l'avancement.\\
A tout moment, on peut donc fournir au client un produit utilisable,
bien que les livraisons soient prévues environ tous les 3 mois. Chaque
trimestre, un module entier et complètement fonctionnel est donc
réceptionné et validé par le client, ce qui permet d'obtenir des
paiements (restons concret tout de même !).




\hypertarget{conditions-de-mise-en-oeuvre}{%
\subsection{\texorpdfstring{\textbf{Conditions de mise en
oeuvre}}{Conditions de mise en oeuvre}}\label{conditions-de-mise-en-oeuvre}}

Extreme Programming apparaît comme la plus radicale des méthodes agiles.
De ce fait, elle draine autour d'elle d'"extrémistes" et farouches
opposants. Parmi eux, le plus actif est sans conteste Matt Stephens qui
critique la méthode dès qu'une occasion se présente (notamment sur plus
d'une trentaine de pages sur internet d'après Google).




\hypertarget{des-uxe9quipes-particuliuxe8res}{%
\subsubsection{Des équipes
particulières}\label{des-uxe9quipes-particuliuxe8res}}

Le facteur humain est très important pour la réussite d'un projet, XP en
est bien conscient...




\hypertarget{une-composition-dexperts}{%
\paragraph{Une composition d'experts}\label{une-composition-dexperts}}

Les buts d'XP sont de palier à des carences constatées dans les projets
informatiques. Cependant, il ne peut y avoir d'évolutions sans
compétences de ses acteurs.\\
C'est pourquoi, les premières limites à l'utilisation de cette méthode
sont les capacités techniques de chaque membre de l'équipe et leurs
capacités relationnelles. En effet, chacun devra être capable d'assumer
une productivité optimale (pendant 35 heures par semaine, soit...)
puisque règne le « death march », c'est à dire une grande conscience
professionnelle pour finir dans les délais le projet.\\
~\\
Aussi, le niveau technique exigé pour un développement avec XP ne
demande pas qu'une bonne connaissance dans le langage de programmation
mais aussi en génie logiciel. En effet, pour construire des choses
solides, il faut avoir mis en place une architecture habile et
puissante.\\
De plus, du fait de la perpétuelle remise en question possible par le
client du produit, clients et développeurs sont disposés à faire évoluer
leurs façons de faire. Cela demande encore une équipe courageuse et très
ouverte.\\
Enfin, avec cette méthode moins structurée, certains développeur,
surtout les plus jeunes. L'absence d'un cadre formel structurant
n'apporte pas l'aide que peut fournir une méthodologie offrant plus de
repères\\
De même, la plupart des responsables peuvent ne pas se sentir à l'aise
ou redouter plus les pépins...\\
Finalement, les membres de l'équipe doivent donc être performants et
polyvalents, ce qui n'est pas forcément innés chez tous les ingénieurs.




\hypertarget{une-taille-duxe9quipe-limituxe9e}{%
\paragraph{Une taille d'équipe
limitée}\label{une-taille-duxe9quipe-limituxe9e}}

Jusqu'à présent, XP n'a été utilisé que par des équipes de 2 à 10
développeurs.\\
Au-delà de 12 personnes, certaines pratiques de XP ne peuvent plus être
adaptées en raison de l'alourdissement des procédures et des problèmes
de communication qui apparaissent. Cela est contraire aux principes même
d'XP !\\
Alors, aussi efficace que la méthode puisse être, elle ne peut
malheureusement pas se généraliser à toutes les organisations et donc à
toutes les entreprises.




\hypertarget{des-experts-en-relationnels}{%
\paragraph{Des experts en
relationnels}\label{des-experts-en-relationnels}}

Avec XP, chacun devra faire preuve d'un bon esprit relationnel pour
discuter et négocier avec le représentant du client, c'est à dire la
personne qui paye !~\\
Aussi, la communication doit être bonne avec l'équipe et plus
particulièrement le binôme, qui change en permanence, ce qui oblige à
travailler avec des personnes de tout âge, ou qu'on n'apprécie pas
forcément...\\
Enfin, il ne faut pas négliger le fait que certains développeurs sont
littéralement "paralysés" dès lors qu'ils se sentent observés ou
seulement moins sérieux...




\hypertarget{pour-des-projets-particuliers}{%
\subsubsection{Pour des projets
particuliers}\label{pour-des-projets-particuliers}}

Le monde de l'informatique est très éclectique, et cela se retrouve dans
les projets...




\hypertarget{le-code-source-de-conflit}{%
\paragraph{Le code, source de conflit
?}\label{le-code-source-de-conflit}}

Avec XP, une classe appartient à une seule et unique personne, ce qui la
responsabilise mais le code de l'application appartient à toute l'équipe
et est donc consultable par tous. Cela a l'avantage de permettre à tous,
une connaissance et une visibilité globale du projet. Chacun est
soucieux de la qualité globale du projet et peut être fier de sa
contribution.~\\
Cependant, si le niveau des développeurs est peu homogène, cela peut
entraîner, du fait de la confiance qu'accorde XP aux développeurs, des
pertes de qualité. Aussi, des critiques peuvent apparaître, voire de
conflits, en raison du refactoring qui permet à chacun, dans un but
d'optimisation, de modifier le code, ce qui peut en vexer plus d'un...\\
XP, par ses exigences, implique aussi une restriction dans le choix des
langages à utiliser. En effet, le choix devra se porter sur un langage
objet qui favorise des itérations courtes et qui permet la mise en œuvre
facile de tests.\\
Parmi ceux correspondant à cette caractéristique, on trouve le
Smalltalk, Python ou Perl, et plus généralement les langages de scripts
mais qui sont de moins en moins enseignés.~\\
Aujourd'hui, ceux qui ont « remportés » le choix des XPiens sont donc le
C++ et le JAVA. Toutefois, le premier est reproché pour sa difficulté de
débuggage (allocation mémoire, erreurs de compilation, d'exécution...)
et commencent d'après certains à faire un peu vieillot par rapport au
miraculeux java. (Pour plus d'info sur les différences en java et le C++
: xpose.free.fr) Avec ces clases JUnit, c'est donc JAVA qui apparaît
comme le plus adapté. D'ailleurs, XP est décrit et conseillé dans le
livre « Thinking in Java ».\\
Mais dans la pratique, tous les développeurs partagent-ils tous cette
préférence ? La culture d'entreprise a-t-elle déjà intégré l'utilisation
de JAVA ?




\hypertarget{pour-des-projets-de-petite-envergure}{%
\paragraph{Pour des projets de petite
envergure}\label{pour-des-projets-de-petite-envergure}}

XP ne peut s'adresser qu'à des projets petits ou moyens en raison de ses
caractéristiques.~\\
En effet, une itération de trois semaines ne semble pas réaliste pour
une équipe de 100 développeurs, trois semaines étant approximativement
le délai pour, tout simplement, contacter et informer l'ensemble des
personnes...~\\
La communication, le feedback, les itérations courtes ne peuvent
s'accommoder à un grand projet.~\\
Aussi, l'investissement de tous est demandé pour un projet XP. Quand
l'équipe s'agrandit cette demande n'est-elle pas trop utopique ?\\
A nouveau, les principes même de XP ne permettent donc pas de
généraliser la méthode à l'ensemble du monde informatique.




\hypertarget{le-client-sur-place}{%
\paragraph{Le client sur place}\label{le-client-sur-place}}

XP demande à l'organisation de pratiquer des tests très peu espacés dans
le temps et en présence d'un représentant du client afin d'avoir un
feedback constant. D'autant plus en raison du mode de données des
spécifications et de l'intégration quotidienne qui obligerait le client
à conserver un de ses responsables pendant des mois avec les
développeurs, sans qu'il soit réellement productif.\\
Ainsi, et c'est probablement un des freins à l'adoption de XP, une
mutation dans le rapport client-développeurs doit avoir lieu pour
respecter les principes de la méthode puisqu'elle suppose une
souscription du client à la plupart des journées de développement.\\
Le client doit donc être intégré à l'équipe de développement, être dans
le même bureau (On-Site Customer), ce qui se marie mal avec le principe
même de la sous-traitance qui est une grosse partie du marché de
l'informatique...





















\section{Les principaux liens en~français~sur l'XP}


\hypertarget{sitographie}{%
\subparagraph[Sitographie]{\texorpdfstring{\protect\hypertarget{Sitographie}{}{}Sitographie}{Sitographie}}\label{sitographie}}

\begin{longtable}[]{@{}l@{}}
\toprule
\endhead
\begin{minipage}[t]{0.97\columnwidth}\raggedright
\protect\hypertarget{Docs}{}{} Cours et documents de référence
\{\#cours-et-documents

-de-référence .western align=``center''\}

\begin{itemize}
\item
  \href{http://citi.insa-lyon.fr/~s\%20frenot/cours/cours01/eXtreme/eXtreme.pdf}{{Cours
  INSA-Lyon (Powerpoint en pdf)}}
\item
  \href{http://www.design-up.com/d\%20ata/presentationxp.pdf}{{Description
  détaillée d'XP avec un retour d'expérience (PDF)}}
\item
  \href{http://www.idealx.org/fr/doc/\%20xp-synthese/}{{Comment mettre
  en pratique XP}}
\item
  \href{http://www.bossavit.com/f\%20iles/ICSSEA.doc}{{"Point de vue
  d'un développeur de la e-économie sur XP" (Word)}}
\item
  \href{http://www.unil.ch/imm/docs\%20/Notes_de_cours/lang_inf/Invites/ExtremeP.pdf}{{Présentation
  d'XP en une page}}
\end{itemize}

\protect\hypertarget{Articles}{}{} Articles

\begin{itemize}
\item
  \href{http://www.01net.com/rdn?o\%20id=151118\&rub=3322\&page=end-151118}{{01net
  : Article très complet de 5 pages}}
\item
  \href{http://developpeur.\%20journaldunet.com/news/010410_extreme.shtml}{{JDNet
  Développeurs : Succinte présentation}}
\item
  \href{http://solutions.jo\%20urnaldunet.com/0105/010509_decrypt_methodes3.shtml}{{JDNet
  (Chronique d'A. Lefèbvre) : Réflexion sur la naissance d'XP et
  présentation}}
\item
  \href{http://www.netsurf.ch/arch\%20ives/2002/02_06/020603nt.html}{{Le
  Nouvel Hebdo : Les principes d'XP}}
\end{itemize}

\protect\hypertarget{Sites}{}{}Sites francophone \{\#sites-francophone
.western align

=``center''\}

\begin{itemize}
\item
  \href{http://www.xp-france.net/}{{Petit site destiné à la communauté
  française d'XP}}
\item
  \href{http://www.gestiondeproj\%20et.com/}{{Un très bon site pour en
  apprendre beaucoup sur la gestion de projets}}
\item
  \href{http://www.afitep.fr/}{{Association Francophone de management de
  projet}}
\end{itemize}

\protect\hypertarget{Divers}{}{}Divers

\begin{itemize}
\tightlist
\item
  \href{http://site.voila.fr/radc\%20p/forumpuma2.PDF}{{Proposition pour
  l'Unification des Méthodes Agiles (PUMA)}}
\end{itemize}\strut
\end{minipage}\tabularnewline
\bottomrule
\end{longtable}

\hypertarget{bibliographie}{%
\subparagraph[
Bibliographie]{\texorpdfstring{\protect\hypertarget{Bibliographie}{}{}
Bibliographie}{ Bibliographie}}\label{bibliographie}}


\begin{longtable}[]{@{}lllll@{}}
\toprule
\endhead
\begin{minipage}[t]{0.17\columnwidth}\raggedright
\href{http://ex\%20tremeprogra\%20mming.free.\%20fr/page.php\%20?page=histo\%20rique}{{[}H{]}\{style=
``text-decor ation: none''\}{istoriqu e}}\strut
\end{minipage} & \begin{minipage}[t]{0.17\columnwidth}\raggedright
\href{http:\%20//extremepr\%20ogramming.f\%20ree.fr/page\%20.php?page=f\%20ondements}{{[}Ses~{]}\{sty
le=``text-de coration: n one''\}{[}f{]}\{st yle=``text-d ecoration:
none''\}{[}onde ments{]}\{styl e=``text-dec oration: no ne''\}}\strut
\end{minipage} & \begin{minipage}[t]{0.17\columnwidth}\raggedright
\href{http\%20://extremep\%20rogramming.\%20free.fr/pag\%20e.php?page=\%20principes}{{[}M{]}\{style=
``text-decor ation: none''\}{[}ise en oeuvre{]}\{sty le=``text-de
coration: n one''\}}\strut
\end{minipage} & \begin{minipage}[t]{0.17\columnwidth}\raggedright
\href{htt\%20p://extreme\%20programming\%20.free.fr/li\%20ens.php}{{[}L{]}\{style=
``text-decor ation: none''\}{[}iens{]}\{st yle=``text-d ecoration:
none''\}}\strut
\end{minipage} & \begin{minipage}[t]{0.17\columnwidth}\raggedright
\href{h\%20ttp://extre\%20meprogrammi\%20ng.free.fr/\%20credits.php}{{[}C{]}\{style=
``text-decor ation: none''\}{rédits}}\strut
\end{minipage}\tabularnewline
\bottomrule
\end{longtable}






\section{Crédits}

Site créé par~Adrien~Machado - Copyright reserved ©

\href{http://www.machado-fr.com/}{{http://www.machado-fr.com}}

\href{mailto:adrien@machado-fr.com}{{adrien@machado-fr.com}}~







\end{document}
